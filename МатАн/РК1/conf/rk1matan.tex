\documentclass[a4paper, 10pt]{article}
\usepackage[T2A]{fontenc}
\usepackage[left=2cm,right=2cm,top=2cm,bottom=2cm]{geometry}
\usepackage[russian]{babel}
\usepackage{amsfonts,amsmath,amssymb}
\usepackage{mathrsfs}
\usepackage{graphicx}
\usepackage[normalem]{ulem}
\usepackage{wrapfig}
\usepackage{fancyhdr}
\usepackage{floatflt}
\usepackage{python}
\usepackage{indentfirst}
\usepackage{setspace}
\usepackage{scrextend}
\usepackage{listings}
\usepackage{makecell,tabularx}
\usepackage{hyperref}
\usepackage{xcolor}

\newcommand{\rub}{{\rm{Р}\kern-.635em\rule[.5ex]{.52em}{.04em}\kern.11em}}

\definecolor{linkcolor}{HTML}{000000} 
\definecolor{urlcolor}{HTML}{0000FF} 

\hypersetup{pdfstartview=FitH,  linkcolor=linkcolor,urlcolor=urlcolor, colorlinks=true}

\definecolor{grey}{RGB}{40, 40, 40}

\renewcommand{\href}[1]{\url{#1}}

\lstdefinestyle{CommentStyle}{
    language=XML,
    %numbers=left, numberstyle=\tiny, stepnumber=1, numbersep=5pt,
    commentstyle=\color{red},
	basicstyle=\footnotesize\ttfamily,
	language={[ANSI]C++},
	keywordstyle=\bfseries,
	showstringspaces=false,
	morekeywords={include, printf},
	commentstyle={},
	escapeinside=§§,
	escapebegin=\begin{russian}\commentfont,
	escapeend=\end{russian},
    keywordstyle=\color{blue}\bfseries,
    morekeywords={align,begin},
    extendedchars=\true,
    tabsize=2
}
\lstdefinestyle{myLatexStyle}{
    language=c++,
    %backgroundcolor=\color{grey},
    numbers=left, numberstyle=\tiny, stepnumber=1, numbersep=5pt,
    commentstyle=\color{red},
    keywordstyle=\color{blue}\bfseries,
    morekeywords={align,begin},
    extendedchars=\true,
    tabsize=2
}

\lstdefinestyle{pmyLatexStyle}{
    language=java,
    %backgroundcolor=\color{grey},
    numbers=left, numberstyle=\tiny, stepnumber=1, numbersep=5pt,
    commentstyle=\color{red},
    keywordstyle=\color{blue}\bfseries,
    morekeywords={align,begin},
    extendedchars=\true,
    tabsize=2
}

\setlength{\parindent}{12,5mm}

\onehalfspacing

\pagestyle{fancy}
\renewcommand{\sectionmark}[1]{\markright{#1}}
\fancyhf{} 
\fancyhead[R]{\bfseries\thepage}
\fancyhead[LO]{\bfseries\rightmark}

\newcommand{\image}[3]{
	\begin{figure}[ht]
		\center{\includegraphics[height=#2pt]{img/#1} }
		\caption{\textit{#3}}\end{figure}
}

\begin{document}

\section{Определения}
    \begin{bf}1. Сформулируйте определение окрестности точки $x \in \mathbb{R}$.\end{bf}

    Окрестностью $U(x_0)$ точки $x_0$ называют любой интервал, содержащий эту точку

    \begin{bf}2. Сформулируйте определение $\epsilon$-окрестности точки $x \in \mathbb{R}$. \end{bf}

    $\epsilon$-окрестностью $U_\epsilon(x_0)$ точки $x_0$ называют интервал с центром в 
    $x_0$ и длиной $2\epsilon$ т.е. $$U_\epsilon(x_0) = (x_0 - \epsilon, x_0 + \epsilon) = 
    \{x \in \mathbb{R},  \epsilon>0 : x_0 - \epsilon < x < x_0 + \epsilon\}$$.

    \begin{bf}3. Сформулируйте определение окрестности $+\infty$. \end{bf}

    Окрестностью точки $+\infty$ называют интервал вида $(a, +\infty)$, где $a$ — произвольное действительное число.
    $$U(+\infty) = \{x \in \mathbb{R}: x > a\}, a > 0$$

    \begin{bf}4. Сформулируйте определение окрестности $-\infty$. \end{bf}

    Окрестностью точки $-\infty$ называют интервал вида $(-\infty, a)$, где $a$ — произвольное действительное число.
    $$U(-\infty) = \{x \in \mathbb{R}: x < a\}, a > 0$$

    \begin{bf}5. Сформулируйте определение окрестности $\infty$. \end{bf}

    Окрестностью бесконечности $\infty$ «без знака»  называют объединение двух бесконечных интервалов 
    $(-\infty, -a) \cup (a, +\infty)$, где a — произвольное действительное число.
    $$U(\infty) = \{x \in \mathbb{R}: |x| > a\}, a > 0$$

    \begin{bf}6. Сформулируйте определение предела последовательности.\end{bf}
    
    Число a называется пределом последовательности $\{x_n\}, n \to +\infty$, если для любого 
    сколь угодно малого $\epsilon > 0$ существует номер $N = N(\epsilon)$ такой, что 
    если порядковый номер члена последовательности $n  \geqslant  N$, то имеет место неравенство 
    $|x_n - a| < \epsilon$. 
    $$\lim\limits_{n \to \infty} x_n = a \iff \ \forall \epsilon > 0 \  \exists N = N(\epsilon) \in 
    \mathbb{N} : \forall n > N(\epsilon) \longrightarrow |x_n - a| < \epsilon$$
    
    \begin{bf}7. Сформулируйте определение сходящейся последовательности.\end{bf}

    Последовательность, придел которой существует и конечен при $n \to \infty$. Поскольку 
    неравенство $|a - x_n| < \epsilon$ эквивалентно неравенству 
    $a - \epsilon < x_n < a + \epsilon$, то все элементы сходящейся последовательности 
    за исключением конечного их числа при любом $\epsilon > 0$ лежат в $\epsilon$-окрестности 
    точки $a$.

    \begin{bf}8. Сформулируйте определение ограниченной последовательности.\end{bf}

    Последовательность ${x_n}$ называется ограниченной снизу, если существует число $c_1$
    такое, что $x_n  \geqslant  c_1$ при всех $n = 1, 2, ...$

    Последовательность ${x_n}$ называется ограниченной сверху, если существует число $c_2$
    такое, что $x_n  \leqslant  c_2$ при всех $n = 1, 2, ...$

    Последовательность ${x_n}$ ограниченная как сверху, так и снизу, называется ограниченной,
    то есть $c_1  \leqslant  x_n  \leqslant  c_2$ при всех $n = 1, 2, ...$

    $$\exists M > 0, M \in \mathbb{R}: \forall n \in \mathbb{N} \longrightarrow |x_n|  \leqslant  M 
    (m  \leqslant  x_n  \leqslant  M, m \in \mathbb{R})$$

    \begin{bf}9. Сформулируйте определение монотонной последовательности.\end{bf}
    
    Последовательность называется монотонной, если она неубывающая($x_{n+1} \geqslant x_n$), возрастающая($x_{n+1}>x_n$), 
    невозрастающая($x_{n+1} \leqslant x_n$) или убывающая($x_{n+1}<x_n$) для $\forall n \in \mathbb{N}$

    \begin{bf}10. Сформулируйте определение возрастающей последовательности.\end{bf}

    Последовательность называется возрастающей, если $x_{n+1} > x_n \forall n \in \mathbb{N}$
        
    \begin{bf}11. Сформулируйте определение убывающей последовательности.\end{bf}

    Последовательность называется убывающей, если $x_{n+1} < x_n \forall n \in \mathbb{N}$
        
    \begin{bf}12. Сформулируйте определение невозрастающей последовательности.\end{bf}
    
    Последовательность называется невозрастающей, если $x_{n+1} \leqslant x_n \forall n \in \mathbb{N}$
        
    \begin{bf}13. Сформулируйте определение неубывающей последовательности.\end{bf}
    
    Последовательность называется неубывающей, если $x_{n+1} \geqslant x_n \forall n \in \mathbb{N}$
    
    \begin{bf}14. Сформулируйте определение фундаментальной последовательности.\end{bf}
    
    Последовательность ${x_n}$ называется фундаментальной, если для любого $\epsilon > 0$ 
    существует номер $N = N(\epsilon)$ такой, что для всех $m  \geqslant  N$ и $n  \geqslant  N$ выполняется 
    неравенство $| x_m - x_n | < \epsilon$.

    \begin{bf}15. Сформулируйте критерий Коши существования предела последовательности.\end{bf}

    Для того, чтобы последовательность была сходящейся, необходимо и достаточно, чтобы она была фундаментальной.

    \begin{bf}16. Сформулируйте определение по Гейне предела функции.\end{bf}

    Пусть функция $f(x)$ определена в проколотой окрестности $\mathring U(x_0)$ точки $x_0$. 
    Число $a$ называется пределом функции $f(x)$ при $x \to x_0$, если для любой последовательности 
    ${x_n}$ точек из $\mathring U(x_0)$ , для которой $\lim\limits_{n \to \infty} x_n = x_0$,
    выполняется равенство $\lim\limits_{n \to \infty} f(x_n) = a$.

    \begin{bf}17. Сформулируйте определение бесконечно малой функции.\end{bf}

    Функция $f(x)$ называется бесконечно малой при $x \to x_0, x_0 \in \mathbb{R}$, если 
    $\lim\limits_{x \to x_0} f(x) = 0$.

    \begin{bf}18. Сформулируйте определение бесконечно большой функции.\end{bf}
    
    Функция $f(x)$ называется бесконечно большой при $x \to x_0, x_0 \in \mathbb{R}$, 
    если $\lim\limits_{x \to x_0} f(x) = +\infty$.

    \begin{bf}19. Сформулируйте определение бесконечно малых функций одного порядка.\end{bf}

    Если существует конечный отличный от нуля предел $\lim\limits_{x \to x_0}\frac{f(x)}{g(x)} = C \ne 0$,
    то говорят, что $f(x)$ и $g(x)$ являются при $x \to x_0$ бесконечно малыми одного порядка и пишут $f(x) = O(g(x))$.

    \begin{bf}20. Сформулируйте определение несравнимых бесконечно малых функций.\end{bf}

    Если при $x \to x_0$ не существует ни конечного, ни бесконечного предела 
    отношения $\frac{f(x)}{g(x)}$ или $\frac{g(x)}{f(x)}$, то говорят, что $f(x)$ и $g(x)$ не сравнимы при $x \to x_0$.

    \begin{bf}21. Сформулируйте определение эквивалентных бесконечно малых функций.\end{bf}
    
    В случае $C = 1$, т.е. если $\lim\limits_{x \to x_0}\frac{f(x)}{g(x)} = 1$, 
    функции $f(x)$ и $g(x)$ называют эквивалентными бесконечно малыми и пишут $f(x) \sim g(x)$, при $x \to x_0$.

    \begin{bf}22. Сформулируйте определение порядка малости одной функции относительно другой.\end{bf}

    Пусть $f(x)$ и $g(x)$ бесконечно малые при $x \to x_0$. Если при некотором $k$
    бесконечно малые $f(x)$ и $(g(x))^k$ являются бесконечно малыми одного порядка,
    то говорят, что $f(x)$ имеет порядок малости $k$ по сравнению с $g(x)$ при $x \to x_0$.

    \begin{bf}23. Сформулируйте определение приращения функции.\end{bf}

    Приращением функции называют $\Delta f(x_0) = f(x) - f(x_0) = f(x_0 + \Delta x) - f(x_0)$.

    \begin{bf}24. Сформулируйте определение непрерывности функции в точке (любое).\end{bf}

    Пусть $X \subset R$, и пусть на $X$ задана числовая функция $f(x)$. Эта функция 
    называется непрерывной в точке $x_0 \in X$, если для любого $\epsilon > 0$ существует 
    число $\delta = \delta(\epsilon) > 0$ такое, что при всех $x, | x - x_0 | < \delta$, 
    выполняется неравенство $| f(x) - f(x_0) | < \epsilon$.

    \begin{bf}25. Сформулируйте определение непрерывности функции на интервале.\end{bf}

    \begin{bf}26. Сформулируйте определение непрерывности функции на отрезке.\end{bf}

    \begin{bf}27. Сформулируйте определение точки разрыва.\end{bf}

    Пусть функция $f(x)$ определена в некоторой окрестности точки $x_0$ или в 
    проколотой окрестности этой точки. Если данная функция не является непрерывной 
    точке $x_0$, то $x_0$ называется точкой разрыва функции $f(x)$.

    \begin{bf}28. Сформулируйте определение точки устранимого разрыва.\end{bf}

    Если $x_0$ — точка разрыва первого рода, и если $f(x_0 - 0) = f(x_0 + 0)$, то такой разрыв называют устранимым.

    \begin{bf}29. Сформулируйте определение точки разрыва I-го рода.\end{bf}
    
    Если $x_0$ — точка разрыва функции $f(x)$, и существуют конечные пределы
    $\lim\limits_{x \to x_0-} f(x) = f(x_0 - 0)$ и $\lim\limits_{x \to x_0+} f(x) = f(x_0 + 0)$,
    то $x_0$ называется точкой разрыва первого рода.

    \begin{bf}30. Сформулируйте определение точки разрыва II-го рода.\end{bf}
    
    Функция $f(x)$ имеет точку разрыва второго рода при $x=x_0$, если по крайней мере один 
    из односторонних пределов не существует или равен бесконечности.

\newpage
\section{Определение предела по Коши}
    \begin{bf}1. Сформулируйте определение по Коши $\lim\limits_{x \to 0} f (x) = b$, где $b \in \mathbb{R}$. Приведите 
    соответствующий пример (с геометрической иллюстрацией).\end{bf}

    Пусть $f(x)$ определена в проколотой окрестности $\mathring U(x_0)$ точки $x_0 = 0$. Число $b \in \mathbb{R}$ 
    называется приделом функции $f(x)$ в точке $x \to x_0$, если для любого $\epsilon > 0$ существует число 
    $N = N(\epsilon) > 0$ такое, что  для всех $x \in \mathring U_N(x_0)$ выполняется неравенство 
    $|f(x) - b| < \epsilon$
    \image{1.png}{200}{$f(x)$}

    \begin{bf}2. Сформулируйте определение по Коши $\lim\limits_{x \to a} f (x) = +\infty$, где $a \in \mathbb{R}$. Приведите 
    соответствующий пример (с геометрической иллюстрацией).\end{bf}

    Пусть $f(x)$ определена в проколотой окресности $\mathring U(x_0), x_0 = a$. Тогда 
    $\lim\limits_{x \to x_0} f(x) = +\infty$, если существует сколь угодно малое $\epsilon > 0$, 
    такое что найдется число $N = N(\epsilon) > 0$, при котором если $x \in \mathring U_N(x_0) \ 
    0 < |x - x_0| < N$, то справедливо неравенство $f(x) > \epsilon$ 
    \image{2.png}{200}{$f(x)$}

    \begin{bf}3. Сформулируйте определение по Коши $\lim\limits_{x \to \infty} f (x) = 0$. Приведите соответствующии 
    пример (с геометрической иллюстрацией).\end{bf}

    Пусть $f(x)$ определена в окресности $+\infty$. Тогда число 0 является пределом функции
    $f(x)$, если для любого сколь угодно малого $\epsilon > 0$ найдется такое $N = N(\epsilon) > 0$, что
    если $x \in \mathring U_N(x_0) \  0 < |x - x_0| < N$, то верно неравенство $|f(x)| < \epsilon$
    \image{3.png}{200}{$f(x)$}

    \begin{bf}4. Сформулируйте определение по Коши $\lim\limits_{x \to a-0} f (x) = -\infty$, где $a \in 
    \mathbb{R}$. Приведите соответствующий пример (с геометрической иллюстрацией).\end{bf}

    $\lim\limits_{x \to a-0} f (x) = -\infty$ - левосторонний предел функции $f(x)$ при $x \to a$, если 
    для любого сколь угодно малого $\epsilon > 0$ найдется такое $N = N(\epsilon) > 0$, что если
    $x \in \mathring U_N^-(x_0)$, то верно неравенство $f(x) < -\epsilon$
    \image{4.png}{200}{$f(x)$}

\newpage
\section{Теоремы}

    \begin{bf}1. Сформулируйте теорему об ограниченности сходящейся числовой последовательности. \end{bf}

    Всякая сходящаяся последовательность ограничена.
    $$\exists \lim\limits_{x \to \infty} x_n = a, a \in \mathbb{R} \rightarrow 
    \exists M > 0: \forall n \in \mathbb{N} \  x_n \in M$$

    \begin{bf}2. Сформулируйте теорему о связи функции, ее предела и бесконечно малой. \end{bf}

    Если существует конечный $\lim\limits_{x \to a} f(x) = A$, то $f(x)$ представлена в виде
    $f(x) = A + \alpha  (x)$, где $\lim\limits_{x \to a} \alpha  (x) = 0$ - бесконечно малая при $x \to a$

    \begin{bf}3. Сформулируйте теорему о сумме конечного числа бесконечно малых функций. \end{bf}

    Сумма конечного числа функций, являющихся бесконечно малыми, при $x \to a$, 
    есть величина бесконечно малая при $x \to a$
    $$\exists \lim\limits_{x \to a} \alpha _k (x) = 0 \rightarrow \lim\limits_{x \to a} 
    \sum\limits_{k=1}^n \alpha _k (x) = 0$$


    \begin{bf}4. Сформулируйте теорему о произведении бесконечно малой на ограниченную функцию. \end{bf}
    
    Если функция $\alpha  (x)$ бесконечно малая при $x \to a$, а $f(x)$ ограниченная функция, то
    $\alpha  (x) \times f(x)$ бесконечно малая, при $x \to a$.

    \begin{bf}5. Сформулируйте теорему о связи бесконечно малой и бесконечно большой функций. \end{bf}

    Если $f(x)$ бесконечно большая при $x \to a$, то $\frac{1}{f(x)}$ бесконечно малая при $x \to a$.
    Если $\alpha  (x)$ бесконечно малая при $x \to a$ и отличная от нуля, то $\frac{1}{\alpha (x)}$ 
    бесконечно большая при $x \to a$

    \begin{bf}6. Сформулируйте теорему о необходимом и достаточном условии эквивалентности бесконечно малых. \end{bf}

    Две бесконечно малые ф-ции при $x \to a$ эквивалентны $\iff$ их разность есть бесконечно
    малая более высокого порядка, чем каждая из них

    $$(f(x) \sim g(x) \ x \to a) \iff (f(x) - g(x) = o(f(x))) \vee (f(x) - g(x) = o(g(x)))$$


    \begin{bf}7. Сформулируйте теорему о сумме бесконечно малых разных порядков. \end{bf}
    
    Если функции $\alpha  (x), \beta (x), ..., \gamma (x)$ бесконечно малые при $x \to a$,
    то $\alpha  (x) + \beta (x) + ... + \gamma (x) \sim \alpha  (x)$ при $x \to a$,
    где $\lim\limits_{x \to a} \frac{\beta (x)}{\alpha (x)} = 0; ...
    \lim\limits_{x \to a} \frac{\gamma (x)}{\alpha (x)} = 0$
    
\end{document}