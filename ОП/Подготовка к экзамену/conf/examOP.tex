\documentclass[a4paper, 10pt]{article}
\usepackage[T2A]{fontenc}
\usepackage[left=2cm,right=2cm,top=2cm,bottom=2cm]{geometry}
\usepackage[russian]{babel}
\usepackage{amsfonts,amsmath,amssymb}
\usepackage{mathrsfs}
\usepackage{graphicx}
\usepackage[normalem]{ulem}
\usepackage{wrapfig}
\usepackage{fancyhdr}
\usepackage{floatflt}
\usepackage{python}
\usepackage{float}
\usepackage{ amssymb }
\usepackage{indentfirst}
\usepackage{setspace}
\usepackage{scrextend}
\usepackage{listings}
\usepackage{makecell,tabularx}
\usepackage{hyperref}
\usepackage{xcolor}

\newcommand{\rub}{{\rm{Р}\kern-.635em\rule[.5ex]{.52em}{.04em}\kern.11em}}

\definecolor{linkcolor}{HTML}{000000} 
\definecolor{urlcolor}{HTML}{0000FF} 

\hypersetup{pdfstartview=FitH,  linkcolor=linkcolor,urlcolor=urlcolor, colorlinks=true}

\definecolor{grey}{RGB}{40, 40, 40}

\renewcommand{\href}[1]{\url{#1}}

\lstdefinestyle{CommentStyle}{
    language=XML,
    %numbers=left, numberstyle=\tiny, stepnumber=1, numbersep=5pt,
    commentstyle=\color{red},
	basicstyle=\footnotesize\ttfamily,
	language={[ANSI]C++},
	keywordstyle=\bfseries,
	showstringspaces=false,
	morekeywords={include, printf},
	commentstyle={},
	escapeinside=§§,
	escapebegin=\begin{russian}\commentfont,
	escapeend=\end{russian},
    keywordstyle=\color{blue}\bfseries,
    morekeywords={align,begin},
    extendedchars=\true,
    tabsize=2
}
\lstdefinestyle{myLatexStyle}{
    language=c++,
    %backgroundcolor=\color{grey},
    numbers=left, numberstyle=\tiny, stepnumber=1, numbersep=5pt,
    commentstyle=\color{red},
    keywordstyle=\color{blue}\bfseries,
    morekeywords={align,begin},
    extendedchars=\true,
    tabsize=2
}

\lstdefinestyle{pmyLatexStyle}{
    language=java,
    %backgroundcolor=\color{grey},
    numbers=left, numberstyle=\tiny, stepnumber=1, numbersep=5pt,
    commentstyle=\color{red},
    keywordstyle=\color{blue}\bfseries,
    morekeywords={align,begin},
    extendedchars=\true,
    tabsize=2
}

\setlength{\parindent}{12,5mm}

\newcommand{\bvec}[1]{\overrightarrow{#1}}
\newcommand{\mcol}[1]{\multicolumn{2}{c}{#1}}
\newcommand{\mcolt}[1]{&#1&}
\renewcommand{\a}{\vec{a}}
\renewcommand{\b}{\vec{b}}
\renewcommand{\c}{\vec{c}}
\renewcommand{\d}{\vec{d}}
\renewcommand{\i}{\vec{i}}
\renewcommand{\j}{\vec{j}}
\renewcommand{\k}{\vec{k}}
\newcommand{\nul}{\vec{0}}

\newcommand{\logo}{\vcenter{\hbox{\includegraphics[width=.8em]{/Users/pluttan/Documents/bw2.png}}}}
\onehalfspacing

\pagestyle{fancy}
\renewcommand{\sectionmark}[1]{\markright{#1}}
\fancyhf{} 
\fancyhead[R]{\bfseries\thepage}
\fancyhead[LO]{$\mathfrak{Copyright}\ \mathfrak{pluttan} \logo$}

\newcommand{\image}[2]{
	\begin{figure}[H]
		\center{\includegraphics[height=#2pt]{img/#1} }
    \end{figure}
}

\newcommand{\dotitle}[3]{

\thispagestyle{empty}
\sloppy{
  \scriptsize{
    \line(6,0){0}

    \centering $\mathfrak{Copyright}\ \mathfrak{pluttan} \logo$

    \centering Привет! Меня зовут Андрей, я создаю свою ботву, этот файл малая ее часть. 
    
    \centering Пользоваться и распространять файлы конечно же можно. Если вы нашли ошибку в файле, можете 
    
    \centering исправить ее в исходном коде и подать на слияние или просто написать в issue. 
    
    \centering Приятного бота)

    \line(6,0){300}

    \centering GitHub: https://github.com/pluttan

    \line(6,0){0}
  }}

\topskip=-200pt
\vspace*{130pt}
\begin{Huge}
  \textbf{
    \begin{center}
        #1
    \end{center}
  }
  {\begin{center} 
        #2
    \end{center}}
\end{Huge}
\vspace*{300pt}
  \begin{flushright}
    Над файлом работали: \\
    pluttan\\
    #3
  \end{flushright}
\newpage
\pagestyle{fancy}
\renewcommand{\sectionmark}[1]{\markright{#1}}
\fancyhf{} 
\fancyhead[R]{\bfseries\thepage}
\fancyhead[LO]{$\mathfrak{Copyright}\ \mathfrak{pluttan} \logo$}
}


% It's XeLaTeX, if you can't compile comment this line 
\usepackage{fontspec}\setmainfont{Times New Roman} 


\begin{document}
\dotitle{Подготовка к экзамену}{Основы программирования}{}
\section{}

\subsection{Синтаксис  и  семантика  языков  программирования.  Алфавит  языка  Pascal.  Описание синтаксиса языка: синтаксические диаграммы. Примеры. }

{\bf{Синтаксис}}
\begin{itemize}
\item правила, определяющие допустимые конструкции языка. «Защищенный» синтаксис предполагает, что предложения языка строятся по правилам, которые позволяют автоматически выявлять большой процент ошибок в программах.
\item совокупность правил, определяющих допустимые конструкции (слова, предложения) языка, его форму
\item правила, определяющие допустимые конструкции языка, построенные из символов его алфавита.  
\end{itemize}

{\bf{Семантика}}
\begin{itemize}
\item правила, определяющие смысл синтаксически корректных предложений. Ясная или «интуитивно-понятная» семантика – семантика, позволяющая без большого труда определять смысл программы или «читать» ее.
\item совокупность правил, определяющих смысл синтаксически корректных конструкций языка, его содержание.
\end{itemize}

{\bf{Алфавит языка Паскаль  }}

\begin{enumerate}
\item латинские буквы без различия строчных и прописных;
\item арабские цифры: 0, 1, 2, 3, 4, 5, 6, 7, 8, 9;
\item шестнадцатеричные цифры: 0..9, а..f или A..F;
\item специальные символы: + - * / = := ; и т. д.;
\item служебные слова: do, while, begin, end и т. д.
\end{enumerate}

{\bf{Описание синтаксиса языка}}

Включает определение алфавита и правил построения различных конструкций языка из символов алфавита и более простых конструкций. Для этого обычно используют либо форму Бэкуса-Наура (БНФ), либо синтаксические диаграммы

Форма Бэкуса-Наура – описание конструкции БНФ состоит из символов алфавита языка, названий более простых конструкций и двух специальных знаков:
\begin{itemize}
\item « ::= » - читаетсяя как “может быть заменено на”
\item « | » - читается как “или”; 
\end{itemize}
В самом описании:

\begin{itemize}
\item Символы алфавита (они же терминальные символы или терминалы) записываются в неизменном виде.
\item Названия конструкций языка (нетерминальные символы или нетерминалы), которые определяются через некоторые другие символы, при записи заключают в угловые скобки (“<..>”).
\end{itemize}
Пример: правило построения конструкции <целое>
\begin{itemize}
    \item <целое> ::= <знак> < целое без знака> | <целое без знака>
    \item <целое без знака> ::= <целое без знака> <цифра> | <цифра>
    \item <цифра> ::= 1 | 2 | 3 | 4 | 5 | 6 | 7 | 8 | 9 | 0
    \item <знак>  ::= + | -
\end{itemize}

{\bf{Синтаксические диаграммы}}
отображают правила построения конструкций в более наглядной форме.
На такой диаграмме:
\begin{itemize}
\item символы алфавита изображают блоками в овальных рамках
\item названия конструкций в прямоугольных рамках
\item правила построения конструкций – в виде линии со стрелками на концах
\item разветвление линии означает, что при построении конструкции есть варианты 
\end{itemize}
Пример: синтаксическая диаграмма, иллюстрирующая первые два правила описания конструкции <целое>
\image{exm1.png}{170}

\subsection{Представление  данных  в  языке  Pascal:  константы  и  переменные.  Классификация  ска-
лярных типов данных, их внутреннее представление, операции над ними. Примеры.} 

Константы – определяются один раз и не изменяются во время выполнения программы.
Используют следующие типы констант:
\begin{itemize}
    \item Целые и вещественные десятичные числа;
    \item Шестнадцатеричные числа;
    \item Логические константы;
    \item Символьные константы – (записываются либо в апострофах, либо в виде соответствующих кодов по таблице ASCII);
    \item Строки символов – записываются в апострофах;
    \item Конструкторы множеств;
    \item Нулевой адрес – nil.
\end{itemize}
Константы используются в двух формах: 
\begin{itemize}
    \item Литерал – представляет собой значение константы, записанное непосредственно в программе;
    \item Поименованные константы объявляются в инструкции раздела описаний const. Обращение к ним осуществляется по имени (идентификатору).
\end{itemize}

\subsection{Совместимость типов данных и операции преобразования типов. Примеры. }



\subsection{Присваивание, условный оператор, оператор выбора. Синтаксис операторов, их особен-
ности и примеры использования. }



\subsection{Операторы циклов языка Pascal. Синтаксис операторов, их особенности и примеры ис-
пользования. }



\subsection{Поисковый цикл. Неструктурная и структурная реализации поискового цикла. }



\subsection{Массивы языка Pascal. Описание, внутреннее представление, операции над массивами и 
их элементами. Примеры. }



\subsection{Строки языка Pascal. Описание, внутреннее представление, операции над строками и их 
элементами. Примеры. }



\subsection{Множества  языка  Pascal.  Описание,  внутреннее  представление,  операции  над  множе-
ствами и их элементами. Примеры. }



\subsection{Записи языка Pascal. Описание, внутреннее представление, операции над записями и их 
элементами. Примеры. }



\subsection{Процедуры и функции. Определение, описание, особенности. Примеры. }



\subsection{Способы передачи данных в подпрограмму на языке Pascal. Примеры. }



\subsection{Локальные и глобальные переменные, законы «видимости» идентификаторов. Примеры. }



\subsection{Формальные и фактические параметры подпрограмм языка Pascal. Примеры. }



\subsection{Параметры-строки, параметры-массивы. Примеры. }



\subsection{Принципы разработки универсальных подпрограмм: «открытые» массивы. Примеры. }



\subsection{Принципы разработки универсальных подпрограмм:  нетипизированные параметры, па-
раметры процедурного типа. Примеры. }



\subsection{Структура модуля языка Pascal. Законы видимости идентификаторов. Доступ к «пере-
крытым» идентификаторам. Примеры. }



\subsection{Рекурсия.  Виды  рекурсии.  Особенности  программирования. Достоинства  и  недостатки. 
Пример. }



\subsection{Адресация динамической памяти: понятие адреса, операции получения адреса и разыме-
нования. Процедуры получения памяти и освобождения ее. Примеры. }



\subsection{Списковые  структуры  данных.  Классификация  и  основные  приемы  работы  с  ними:  соз-
дание элемента, добавление элемента к списку, удаление элемента из списка. Область приме-
нения списковых структур данных. Пример. }



\subsection{Основы файловой системы: файл, каталог, полное имя файла, внутреннее представление 
информации в файле. Файловая переменная. Операции открытия и закрытия файлов. Приме-
ры. }



\subsection{Текстовые  файлы.  Внутреннее  представление  информации  в  файле.  Операции  над  фай-
лами. Пример. }



\subsection{Типизированные  файлы:  внутреннее  представление  информации  в  файле.  Операции  над 
файлами. Пример. }



\subsection{Нетипизированные  файлы.  Внутреннее  представление  информации  в  файле.  Операции 
над файлами. Пример. }



\subsection{Классы консольного режима среды Lazarus: описание классов, поля и методы, объявление 
объектов класса, доступ к полям и методам объекта, ограничение доступа. Пример. }



\subsection{Классы консольного режима  среды  Lazarus: Способы инициализация полей. Неявный па-
раметр Self. Пример. }



\subsection{Процедурная и объектная декомпозиция. Диаграммы классов. Отношения между класса-
ми. Примеры. }



\subsection{Динамические объекты и объекты с динамическими полями в консольном режиме среды 
Lazarus. Примеры. }



\subsection{Технология  событийного  программирования.  События  операционной  системы,  сообще-
ния и события Lazarus. Основные события Lazarus. Примеры.}



\end{document}