\documentclass[a4paper, 10pt]{article}
\usepackage[T2A]{fontenc}
\usepackage[left=2cm,right=2cm,top=2cm,bottom=2cm]{geometry}
\usepackage[russian]{babel}
\usepackage{amsfonts,amsmath,amssymb}
\usepackage{mathrsfs}
\usepackage{graphicx}
\usepackage[normalem]{ulem}
\usepackage{wrapfig}
\usepackage{fancyhdr}
\usepackage{floatflt}
\usepackage{python}
\usepackage{float}
\usepackage{ amssymb }
\usepackage{indentfirst}
\usepackage{setspace}
\usepackage{scrextend}
\usepackage{listings}
\usepackage{makecell,tabularx}
\usepackage{hyperref}
\usepackage{xcolor}

\newcommand{\rub}{{\rm{Р}\kern-.635em\rule[.5ex]{.52em}{.04em}\kern.11em}}

\definecolor{linkcolor}{HTML}{000000} 
\definecolor{urlcolor}{HTML}{0000FF} 

\hypersetup{pdfstartview=FitH,  linkcolor=linkcolor,urlcolor=urlcolor, colorlinks=true}

\definecolor{grey}{RGB}{40, 40, 40}

\renewcommand{\href}[1]{\url{#1}}

\lstdefinestyle{CommentStyle}{
    language=XML,
    %numbers=left, numberstyle=\tiny, stepnumber=1, numbersep=5pt,
    commentstyle=\color{red},
	basicstyle=\footnotesize\ttfamily,
	language={[ANSI]C++},
	keywordstyle=\bfseries,
	showstringspaces=false,
	morekeywords={include, printf},
	commentstyle={},
	escapeinside=§§,
	escapebegin=\begin{russian}\commentfont,
	escapeend=\end{russian},
    keywordstyle=\color{blue}\bfseries,
    morekeywords={align,begin},
    extendedchars=\true,
    tabsize=2
}
\lstdefinestyle{myLatexStyle}{
    language=c++,
    %backgroundcolor=\color{grey},
    numbers=left, numberstyle=\tiny, stepnumber=1, numbersep=5pt,
    commentstyle=\color{red},
    keywordstyle=\color{blue}\bfseries,
    morekeywords={align,begin},
    extendedchars=\true,
    tabsize=2
}

\lstdefinestyle{pmyLatexStyle}{
    language=java,
    %backgroundcolor=\color{grey},
    numbers=left, numberstyle=\tiny, stepnumber=1, numbersep=5pt,
    commentstyle=\color{red},
    keywordstyle=\color{blue}\bfseries,
    morekeywords={align,begin},
    extendedchars=\true,
    tabsize=2
}

\setlength{\parindent}{12,5mm}

\newcommand{\bvec}[1]{\overrightarrow{#1}}
\newcommand{\mcol}[1]{\multicolumn{2}{c}{#1}}
\newcommand{\mcolt}[1]{&#1&}
\renewcommand{\a}{\vec{a}}
\renewcommand{\b}{\vec{b}}
\renewcommand{\c}{\vec{c}}
\renewcommand{\d}{\vec{d}}
\renewcommand{\i}{\vec{i}}
\renewcommand{\j}{\vec{j}}
\renewcommand{\k}{\vec{k}}
\newcommand{\nul}{\vec{0}}

\newcommand{\logo}{\vcenter{\hbox{\includegraphics[width=.8em]{/Users/pluttan/Documents/bw2.png}}}}
\onehalfspacing

\pagestyle{fancy}
\renewcommand{\sectionmark}[1]{\markright{#1}}
\fancyhf{} 
\fancyhead[R]{\bfseries\thepage}
\fancyhead[LO]{$\mathfrak{Copyright}\ \mathfrak{pluttan} \logo$}

\newcommand{\image}[2]{
	\begin{figure}[H]
		\center{\includegraphics[height=#2pt]{img/#1} }
    \end{figure}
}
% It's XeLaTeX, if you can't compile:
% 1. download & install this https://isopromat.ru/wp-content/uploads/fonts-GOST.zip
% 2. try to use XeLaTeX 
% OR comment this line 
\usepackage{fontspec}\setmainfont{GOST type B} 

\begin{document}
\section{Теория}
\subsection{Свойства прямоугольного проецирования}

Привет! Я текст)
$y = x$

\subsection{Какие линии называются проецирующими линиями, линиями уровня?}
\subsection{Какие линии, принадлежащие плоскости, называются горизонталью, фронталью?}
\subsection{Теорема о проецировании прямого угла}
\subsection{На основании каких положений строят перпендикулярные: прямую и плоскость?}
\subsection{На основании каких положений строят параллельные: прямую и плоскость?}
\subsection{На основании каких положений строят на чертеже две параллельные плоскости?}
\subsection{На основании каких положений строят на чертеже две перпендикулярные плоскости?}
\subsection{Правило построения проекции точки, принадлежащей поверхности}
\subsection{Правило построения проекции точки, принадлежащей плоскости}
\subsection{Правило построения проекций точки, принадлежащей поверхности вращения}
\subsection{Способы преобразования}
\subsection{Условия преобразования способом замены плоскостей проекций}
\subsection{Условия преобразования способом вращения вокруг проецирующей прямой}
\subsection{Какая линия поверхности вращения называется меридианом, параллелью?}
\subsection{В какую линию может проецироваться окружность при разных ее положениях отностельно плоскостей проекций?}
\subsection{Алгоритм построения точек пересечения линии с поверхностью}
\subsection{Последовательность построения точки пересечения прямой и плоскости}
\subsection{Последовательность построения точек пересечения прямой и поверхности}
\subsection{Какие линиии получаются в сечении цилиндрической поверхности плоскостью при разных положениях плоскости относительно оси цилиндрической поверхности?}
\subsection{Конические сечения. При каком положении плоскости относительно оси конической поверхности сечением является окружность, эллипс, прямые, парабола, гипербола?}
\subsection{Последовательность построения линии пересечения двух поверхностей}
\subsection{Теорема Монжа. Привести пример}
\subsection{Какую плоскость называют касательной к поверхности в данной точке?}
\subsection{Что называется нормалью к поверхности в данной точке?}

\end{document}