
\vspace*{15pt}

\df{Верхняя треугольная матрица}{квадратная матрица, все элементы под главной диагональю которой равны 0.}

\vspace*{15pt}

\df{Нижняя треугольная матрица}{квадратная матрица, все элементы над главной диагональю которой равны 0.}

\subsection{Дать определение равенства матриц.}

Матрицы называются {\bf{равными}}, если:
\begin{enumerate}
    \item[1)] они имеют одинаковый тип,
    \item[2)] У них совпадают все соответствующие элементы.
\end{enumerate}

 \begin{center} 
    Для $A = (a_{ij})$ и $B = (b_{ij})$
 
    $A = B \iff A, B \in M_{mn}(\mathbb{R})$ и $a_{ij} = b_{ij}$ $\forall ij$  
\end{center}



\subsection{Дать определение суммы матриц и произведения матрицы на число.}

\df{Сумма матриц $A = (a_{ij})$ и $B = (b_{ij})$ одного типа $m\times n$}{матрица $C = (c_{ij})$ того же типа $m\times n$ с элементами $c_{ij} = a_{ij} + b_{ij}$.}

\vspace*{15pt}

\df{Произведение матрицы $A = (a_{ij})$ типа $m\times n$ на число $\alpha \in \mathbb{R}$}{матрица $C = (c_{ij})$ того же типа $m\times n$ с элементами $c_{ij} = \alpha a_{ij}$.}

\subsection{Дать определение операции транспонирования матриц.}

\dft{Для матрицы $A = (a_{ij})$ типа $m\times n$ }{ее транспонированной матрицей}{называется матрица $A^T = (c_{ij})$ типа $n\times m$ с элементами $c_{ij} = a_{ji}$}

При транспонировании матрицы ее строки (столбцы) страновятся столбцами (строками) с теми же номерами.

\subsection{Дать определение операции умножения матриц.}

\dfn{Произведением матрицы $A = (a_{ij})$ типа $m\times n$ и матрицы $B = (b_{ij})$ типа $n\times p$}{называется матрица $C = (c_{ij})$ типа $m \times p$ с элементами $c_{ij} = \overset{n}{\underset{k = 1}{\sum}}a_{ik}b_{kj} = a_{i1}b_{1j}+...+a_{in}b_{nj}$.}

$$
\begin{pmatrix} 
    a_{11}&a_{12}&\ldots&a_{1n}\\ 
    \vdots&\vdots&\ddots&\vdots\\
    \mathbf{a_{i1}}&\mathbf{a_{i2}}&\mathbf{\ldots}&\mathbf{a_{in}}\\ 
    \vdots&\vdots&\ddots&\vdots\\
    a_{m1}&a_{m2}&\ldots&a_{mn}\\ 
\end{pmatrix}
\times
\begin{pmatrix} 
    b_{11}&\ldots&\mathbf{b_{1j}}&\ldots&b_{1p}\\ 
    b_{21}&\ldots&\mathbf{b_{2j}}&\ldots&b_{2p}\\ 
    \vdots&\ddots&\mathbf{\vdots}&\ddots&\vdots\\
    b_{n1}&\ldots&\mathbf{b_{nj}}&\ldots&b_{np}\\ 
\end{pmatrix}
=
\begin{pmatrix} 
    c_{11}&c_{12}&\ldots&c_{1j}&\ldots&c_{1p}\\ 
    c_{21}&c_{22}&\ldots&c_{2j}&\ldots&c_{2p}\\ 
    \vdots&\vdots&\ddots&\vdots&\ddots&\vdots\\
    c_{i1}&c_{i2}&\ldots&\mathbf{c_{ij}}&\ldots&c_{ip}\\ 
    \vdots&\vdots&\ddots&\vdots&\ddots&\vdots\\
    c_{m1}&c_{m2}&\ldots&c_{mj}&\ldots&c_{mp}\\ 
\end{pmatrix}
$$

$AB \ne BA$ (как правило).

\subsection{Дать определение обратной матрицы.}

\dft{Пусть $A$ - квадратная матрица порядка $n$. Матрица $B$ называется}{обратной}{к матрице $A$, если:
\begin{enumerate}
    \item Она того же порядка $n$,
    \item $AB = BA = E$, где $E$ - единичная матрица.
\end{enumerate}
}

\subsection{Дать определение минора. Какие миноры называются окаймляющими для данного минора матрицы?}

{\bf{Минором}} порядка $k$ матрицы $A$ типа $m\times n$ называется определитель, который составлен из элементов этой матрицы, стоящих на пересечении любых $k$ строк и $k$ столбцов с сохранением порядка этих строк и столбцов.

Обозначение: минор $M^{j_1 ... j_k}_{i_1 ... i_k}$ составлен из элементов, расположенных на пересечении строк $i_1, ..., i_k$ и столбцов $j_1, ..., j_k$, причем $i_1<...<i_k$, $j_1<...<j_k$.

\vspace*{15pt}

Минор $M'$ матрицы $A$ называется {\bf{окаймляющим}} для минора $M$, если он получается из $M$ добавлением одной новой строки и одного нового столбца, причем эти строка и столбец входят в матрицу $A$ и не входят в минор $M$.

\subsection{Дать определение базисного минора и ранга матрицы.}

\df{Ранг матрицы}{число, равное максимальному проядку среди ее ненулевых миноров.}

\vspace*{15pt}

\dft{Минор $M$ матрицы $A$ называется}{базисным,}{если
\begin{enumerate}
    \item[1)] он не равен нулю,
    \item[2)] его порядок равен $RgA$.
\end{enumerate}
}

У матрицы может быть несколько базисных миноров.

\subsection{Дать определение однородной и неоднородной СЛАУ.}

\dfn{Системой линейных алгебраических уравнений}{называется система вида 
$$
\begin{cases}
    a_{11}x_1+a_{12}x_2+\ldots+a_{1n}x_n = b_1\\
    a_{21}x_1+a_{22}x_2+\ldots+a_{2n}x_n = b_2\\
    \ldots\ldots\ldots\ldots\ldots\ldots\ldots\ldots\ldots\ldots\ldots\\
    a_{m1}x_1+a_{m2}x_2+\ldots+a_{mn}x_n = b_m
\end{cases}
$$
Где $a_{ij}, b_i, x_i \in \mathbb{R}$
}

Числа $a_{ij}$ называются коэффициентами системы,
$b_{ij}$ называется свободными членами.

СЛАУ называется {\bf{однородной}}, если все $b$ равны $0$, {\bf{неоднородной}}, если хотя бы один из $b_i$ не равен $0$.

\subsection{Дать определение фундаментальной системы решений однородной СЛАУ.}

Пусть дана однородная СЛАУ $AX = \Theta$ с $n$ неизвесными $x_1, ..., x_n$, и пусть $RgA = r$. Фундаментальной системой решений (ФСР) однородной СЛАУ $AX = \Theta$ называется любой набор из $k = n - r$ линейно независимых столбцов $x^{(1)}, ..., x^{(k)}$ является решениями этой системы.

\subsection{Записать формулы для нахождения обратной матрицы к произведению двух обратимых матриц и для транспонированной матрицы.}

Обратная матрица к произведению двух обратимых матриц: если квадратные матрицы $A$ и $B$ одного порядка и имеют обратные матрицы $A^{-1}$ и $B^{-1}$, то их произведение $AB$ имеет обратную матрицу $AB^{-1}$, причем $(AB)^{-1} = B^{-1}A^{-1}$.

\vspace*{15pt}

Обратная матрица для транспонированной матрицы: если квадратная матрица $A$ имеет обратную матрицу $A^{-1}$, то транспонированная матрица $A^T$ тоже имеет обратную матрицу $(A^T)^{-1}$, причем $(A^T)^{-1} = (A^{-1})^T$.

\subsection{Дать определение присоединённой матрицы и записать формулу для вычисления обратной матрицы.}

{\bf{Присоединеной матрицей}} для квадратной матрицы $A$ называется матрица $A^* = (A_{ji})$, где $(A_{ij})$ - матрица из алгебраических дополнений для соответствующих элементов.

Формула для вычисления обратной матрицы
$$A^{-1} = \frac{1}{detA}A^*$$

%note : искуственный переход на новую страницу
\vspace*{15pt}
\vspace*{15pt}

\subsection{Перечислить элементарные преобразования матриц.}

Элементарные преобразования матриц
\begin{enumerate}
    \item[1)] Умножение строки (столбца) матрицы на число $\lambda \ne 0$:
    \item[2)] Перестановка двух строк (столбцов).
    \item[2)] Добавление к одной строке (столбцу) матрицы другой строки (столбца), умноженной на число.
\end{enumerate}

\subsection{Записать формулы Крамера для решения системы линейных уравнений с обратимой матрицей.}

СЛАУ $AX = B$, где $A$ - квадратная и $detA \ne 0$, имеет единственное решение, причем $x_1 = \frac{\Delta_1}{\Delta}, ..., x_n = \frac{\Delta_n}{\Delta}$, где $\Delta = detA$,
$$\Delta_1 = 
\begin{vmatrix}
    b_1&a_{12}&\ldots&a_{1n}\\
    \vdots&\vdots&\ddots&\vdots\\
    b_n&a_{n2}&\ldots&a_{nn}\\
\end{vmatrix}, ..., 
\Delta_n = 
\begin{vmatrix}
    a_{11}&\ldots&a_{1n-1}&b_1\\
    \vdots&\ddots&\vdots&\vdots\\
    a_{n1}&\ldots&a_{nn-1}&b_n\\
\end{vmatrix}
$$

\subsection{Перечислить различные формы записи системы линейных алгебраических уравнений (СЛАУ). Какая СЛАУ называется совместной?}

\noindent
Формы записи СЛАУ:
\begin{enumerate}
    \item Координатная: 
    $$
    \begin{cases}
        a_{11}x_1+a_{12}x_2+\ldots+a_{1n}x_n = b_1\\
        a_{21}x_1+a_{22}x_2+\ldots+a_{2n}x_n = b_2\\
        \ldots\ldots\ldots\ldots\ldots\ldots\ldots\ldots\ldots\ldots\ldots\\
        a_{m1}x_1+a_{m2}x_2+\ldots+a_{mn}x_n = b_m
    \end{cases}
    a_{ij}, b_i, x_i \in \mathbb{R}
    $$
    \item Векторная:
    $$x_1
    \begin{pmatrix}
        a_{11}\\a_{21}\\\ldots\\a_{m1}
    \end{pmatrix}+x_2
    \begin{pmatrix}
        a_{12}\\a_{22}\\\ldots\\a_{m2}
    \end{pmatrix}+...+x_n
    \begin{pmatrix}
        a_{1n}\\a_{2n}\\\ldots\\a_{mn}
    \end{pmatrix}=
    \begin{pmatrix}
        b_{1}\\b_{2}\\\ldots\\b_{m}
    \end{pmatrix}
    $$
    \begin{center} или \end{center}
    $$
    x_1\vec{a_1}+x_2\vec{a_2}+...+x_n\vec{a_n} = \vec{b}
    $$
    \item Матричная:
    $$
    \begin{pmatrix}
        a_{11}&a_{12}&\ldots&a_{1n}\\
        a_{21}&a_{22}&\ldots&a_{2n}\\
        \vdots&\vdots&\ddots&\vdots\\
        a_{m1}&a_{m2}&\ldots&a_{mn}
    \end{pmatrix}
    \begin{pmatrix}
        x_{1}\\x_{2}\\\vdots\\x_{m}
    \end{pmatrix} = 
    \begin{pmatrix}
        b_{1}\\b_{2}\\\vdots\\b_{m}
    \end{pmatrix}
    $$
    \begin{center} или \end{center}
    $$
    AX = B \text{ } (A\vec{x} = \vec{b})
    $$
    
\end{enumerate}

СЛАУ называется совместной (несовместной), если она имеет (не имеет) решение.

\subsection{Привести пример, показывающий, что умножение матриц некоммутативно.}

Некомутативность произведение матриц: $AB\ne BA$ (как правило, но бывают исключения)

$A = \begin{pmatrix}1&0\\0&0\end{pmatrix}, 
           B = \begin{pmatrix}0&1\\0&0\end{pmatrix}$ :  $
           AB = \begin{pmatrix}0&1\\0&0\end{pmatrix},
           BA = \begin{pmatrix}0&0\\0&0\end{pmatrix}
           \Rightarrow AB \ne BA$