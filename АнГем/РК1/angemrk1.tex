\documentclass[a4paper, 10pt]{article}
\usepackage[T2A]{fontenc}
\usepackage[left=2cm,right=2cm,top=2cm,bottom=2cm]{geometry}
\usepackage[russian]{babel}
\usepackage{amsfonts,amsmath,amssymb}
\usepackage{mathrsfs}
\usepackage{graphicx}
\usepackage[normalem]{ulem}
\usepackage{wrapfig}
\usepackage{fancyhdr}
\usepackage{floatflt}
\usepackage{python}
\usepackage{float}
\usepackage{ amssymb }
\usepackage{indentfirst}
\usepackage{setspace}
\usepackage{scrextend}
\usepackage{listings}
\usepackage{makecell,tabularx}
\usepackage{hyperref}
\usepackage{xcolor}

\newcommand{\rub}{{\rm{Р}\kern-.635em\rule[.5ex]{.52em}{.04em}\kern.11em}}

\definecolor{linkcolor}{HTML}{000000} 
\definecolor{urlcolor}{HTML}{0000FF} 

\hypersetup{pdfstartview=FitH,  linkcolor=linkcolor,urlcolor=urlcolor, colorlinks=true}

\definecolor{grey}{RGB}{40, 40, 40}

\renewcommand{\href}[1]{\url{#1}}

\lstdefinestyle{CommentStyle}{
    language=XML,
    %numbers=left, numberstyle=\tiny, stepnumber=1, numbersep=5pt,
    commentstyle=\color{red},
	basicstyle=\footnotesize\ttfamily,
	language={[ANSI]C++},
	keywordstyle=\bfseries,
	showstringspaces=false,
	morekeywords={include, printf},
	commentstyle={},
	escapeinside=§§,
	escapebegin=\begin{russian}\commentfont,
	escapeend=\end{russian},
    keywordstyle=\color{blue}\bfseries,
    morekeywords={align,begin},
    extendedchars=\true,
    tabsize=2
}
\lstdefinestyle{myLatexStyle}{
    language=c++,
    %backgroundcolor=\color{grey},
    numbers=left, numberstyle=\tiny, stepnumber=1, numbersep=5pt,
    commentstyle=\color{red},
    keywordstyle=\color{blue}\bfseries,
    morekeywords={align,begin},
    extendedchars=\true,
    tabsize=2
}

\lstdefinestyle{pmyLatexStyle}{
    language=java,
    %backgroundcolor=\color{grey},
    numbers=left, numberstyle=\tiny, stepnumber=1, numbersep=5pt,
    commentstyle=\color{red},
    keywordstyle=\color{blue}\bfseries,
    morekeywords={align,begin},
    extendedchars=\true,
    tabsize=2
}

\setlength{\parindent}{12,5mm}

\newcommand{\bvec}[1]{\overrightarrow{#1}}
\newcommand{\mcol}[1]{\multicolumn{2}{c}{#1}}
\newcommand{\mcolt}[1]{&#1&}

\newcommand{\logo}{\vcenter{\hbox{\includegraphics[width=.8em]{/Users/pluttan/Documents/bw2.png}}}}
\onehalfspacing

\pagestyle{fancy}
\renewcommand{\sectionmark}[1]{\markright{#1}}
\fancyhf{} 
\fancyhead[R]{\bfseries\thepage}
\fancyhead[LO]{$\mathfrak{Copyright}\ \mathfrak{pluttan} \logo$}

\newcommand{\image}[2]{
	\begin{figure}[H]
		\center{\includegraphics[height=#2pt]{img/#1} }
    \end{figure}
}

\begin{document}

\section{Базовые теоретические вопросы}

\subsection{Дать определение равенства геометрических векторов.}

Два вектора называются равными, если они сонаправлены и их длины равны.

\subsection{Дать определения суммы векторов и произведения вектора на число.}

\begin{center}
\begin{tabular}{c c} 
    \mcol{ Суммой двух векторов $\vec{a}$ и $\vec{b}$ называется такой вектор $\vec{c}$, 
    построенный по следующим правилам}\\
    Правило парралелограмма & Правило треугольника\\
    \mcol{Пусть $O$ любая точка.}\\
    \mcol{Отложим $\vec{a}$ от $O$. Получим $\bvec{OA}$.}\\
    \mcol{Отложим $\vec{b}$ от}\\
    Точки $O$ & Точки $A$\\
    \mcol{Получим}\\
\begin{minipage}[t]{0.4\linewidth}\image{1.png}{100}\end{minipage}&
\begin{minipage}[t]{0.4\linewidth}\image{2.png}{100}\end{minipage}\\
    \mcol{Построим}\\
    Парралелограмм & Треугольник\\
    \mcol{Вектор $\vec{c}$, представителем которого является $\bvec{OC}$ - искомый.}\\
    \mcol{Построение не зависит от выбора точки $O$ и правила построения.}\\\\
\end{tabular}
\end{center}

Произведением $\vec{a}$ на число $\alpha$ называется $\vec{b}$, если он коллинеарен $\vec{a}$
(причем если $\vec{a} \upuparrows \vec{b}: \ \alpha > 0$, иначе $\alpha < 0$ ) и его 
длина равна $|\alpha| |\vec{a}|$ 

\subsection{Дать определения коллинеарных и компланарных векторов.}

\begin{center}
\begin{tabular}{c c} 
    \mcol{Геометрические вектора называются}\\
    Коллинеарными & Комплонарными\\
    \mcol{Если они лежат }\\
    На одной или парралельных прямых & На одной или парралельных плоскостях \\
\end{tabular}
\end{center}

\subsection{Дать определение линейно зависимой и линейно независимой системы векторов.}

\begin{center}
\begin{tabular}{c c} 
    \mcol{Векторы $\vec{a_1},...,\vec{a_n}$ назваются линейно}\\
    Зависимыми & Независимыми\\
    Если существует & Если не существует\\
    \mcol{Их нетривиальная линейная комбинация, равная $\vec{0}$, т.е.}\\
    \mcol{Если при $\alpha_1,...,\alpha_n \in \mathbb{R}, \ \alpha_1 \vec{a_1} +...+ \alpha_n \vec{a_n} = 0$ }\\
    $\exists \alpha_1,...,\alpha_n$ отличные от нуля & $\nexists \alpha_1,...,\alpha_n$ отличные от нуля \\
\end{tabular}
\end{center}

\subsection{Сформулировать геометрические критерии линейной зависимости 2-х и 3-х векторов.}

2 вектора линейно зависимы $\iff$ они коллинеарны

3 вектора линейно зависимы $\iff$ они комплонарны

\subsection{Дать определение базиса и координат вектора.}

\begin{center}
\begin{tabular}{c c c} 
    \mcolt{Базисом в пространстве}\\
    $V_1$&$V_2$&$V_3$\\
    \mcolt{Называется}\\
    Любой ненулевой & Любая упорядоченная пара & Любая упорядоченная тройка\\
    вектор & коллинеарных векторов & некомплонарных векторов\\
    $\forall \vec{x} \in V_1 \exists ! x \in \mathbb{R}$&
    $\forall \vec{x} \in V_2 \exists ! x_1, x_2 \in \mathbb{R}$&
    $\forall \vec{x} \in V_3 \exists ! x_1, x_2, x_3 \in \mathbb{R}$\\
    $\vec{x} = x\vec{e}$ & $\vec{x} = x_1\vec{e_1} + x_2\vec{e_2}$ &
    $\vec{x} = x_1\vec{e_1} + x_2\vec{e_2} + x_3\vec{e_3}$\\
    \mcolt{Коэффициенты разложения}\\
    $x$&$x_1, x_2$&$x_1, x_2, x_3$\\
    \mcolt{Называются координатами $\vec{x}$ в базисе}\\
    $\vec{e}$&$\vec{e}_1, \vec{e}_2$&$\vec{e}_1, \vec{e}_2, \vec{e}_3$\\
\end{tabular}
\end{center}

\subsection{Сформулировать теорему о разложении вектора по базису.}

\begin{center}
Любой вектор можно разложить по базису, причем единственным способом
\begin{tabular}{c c c} 
    $\forall \vec{x} \in V_1 \exists ! x \in \mathbb{R}$&
    $\forall \vec{x} \in V_2 \exists ! x_1, x_2 \in \mathbb{R}$&
    $forall \vec{x} \in V_3 \exists ! x_1, x_2, x_3 \in \mathbb{R}$\\
\end{tabular}
\end{center}

\subsection{Дать определение ортогональной скалярной проекции вектора на направление.}

Ортогональной проекцией вектора $\vec{a}$ на вектор $\vec{b}$ называется число, вычисленное по правилу:
\begin{enumerate}
    \item Отложим вектор $\vec{a}$ от любой точки $A$, получим $\bvec{AB}$
    \item Возьмем любую ось $b$, направление которой совпадает с $\vec{b}$
    \item Спроецируем $\bvec{AB}$ на $b$ и получим  $\bvec{A_{np}B_{np}}$
    \item Найдем число $\pm |\bvec{A_{np}B_{np}}|$, где $+$ если $\bvec{A_{np}B_{np}} \upuparrows \vec{b}$, иначе $-$
\end{enumerate}
Обозначение $np_{\vec{b}} \vec{a}$

\subsection{Дать определение скалярного произведения векторов.}

Скалярным произведением 2 векторов $\vec{a}$ и $\vec{b}$ называется число, равное произведению длин
векторов на косинус угла между ними.

\subsection{Сформулировать свойство линейности скалярного произведения.}

\begin{center}
$\forall \vec{a}, \vec{b} \forall k \in \mathbb{R}$

$(k\vec{a})\vec{b} = k(\vec{a}\vec{b})$

$\vec{a}(k\vec{b}) = k(\vec{a}\vec{b})$\\

$\forall \vec{a}, \vec{b}, \vec{c}$

$(\vec{a} + \vec{b})\vec{c} = \vec{a}\vec{c} + \vec{b}\vec{c}$

$\vec{a}(\vec{b} + \vec{c}) = \vec{a}\vec{b} + \vec{a}\vec{c}$
\end{center}
\subsection{Записать формулу для вычисления скалярного произведения двух векторов, заданных в ортонормированном базисе.}

$$\vec{a}\vec{b} = \vec{a}_1\vec{b}_1 + \vec{a}_2\vec{b}_2 + \vec{a}_3\vec{b}_3,
\vec{a}\{a_1;a_2;a_3\},\vec{b}\{b_1;b_2;b_3\} $$

\subsection{Записать формулу для вычисления косинуса угла между векторами, заданными в ортонормированном базисе.}

$$cos(\vec{a},\vec{b}) = \frac{ \vec{a} \vec{b} }{ |\vec{a}| |\vec{b}| } = 
\frac{\vec{a}_1\vec{b}_1 + \vec{a}_2\vec{b}_2 + \vec{a}_3\vec{b}_3}
{\sqrt{\vec{a}_1 + \vec{a}_2 + \vec{a}_3}\sqrt{\vec{b}_1+\vec{b}_2+\vec{b}_3}}$$

\subsection{Дать определение правой и левой тройки векторов.}

\begin{center}
\begin{tabular}{c c} 

    \mcol{Упорядоченная тройка некомплонарных векторов $\vec{a}, \vec{b}, \vec{с}$ называется }\\
    Правой & Левой \\
    \mcol{Если кратчайший поворот от $\vec{a}$ к $\vec{b}$ виден из конца $\vec{c}$ проходящей}\\
    Против часовой стрелке & По часовой стрелке\\

\end{tabular}
\end{center}

\subsection{Дать определение векторного произведения векторов.}

Векторным произведением 2 векторов $\vec{a}$, $\vec{b}$ называется вектор $\vec{c}$, удовлетворяющих условиям
\begin{enumerate}
    \item $\vec{c} \perp \vec{a}, \vec{c} \perp \vec{b}$
    \item Упорядоченная тройка $\vec{a}$, $\vec{b}$, $\vec{c}$ правая
    \item $|\vec{c}| = |\vec{a}||\vec{b}|sin(\vec{a}, \vec{b})$
\end{enumerate}

\subsection{Сформулировать свойство коммутативности (симметричности) скалярного произведения 
и свойство антикоммутативности (антисимметричности) векторного произведения.}

\begin{center}
$\forall \vec{a}, \vec{b}$

$\vec{a}\vec{b} = \vec{b}\vec{a}$ - симметричность скалярного произведения

$\vec{a} \times \vec{b} = - \vec{b} \times \vec{a}$ - кососимметричность векторного произведения
\end{center}

\subsection{Сформулировать свойство линейности векторного произведения векторов.}

\begin{center}
    $\forall \vec{a}, \vec{b} \forall k \in \mathbb{R}$
    
    $(k\vec{a}) \times \vec{b} = k(\vec{a} \times \vec{b})$
    
    $\vec{a} \times (k\vec{b}) = k(\vec{a} \times \vec{b})$\\
    
    $\forall \vec{a}, \vec{b}, \vec{c}$
    
    $(\vec{a} + \vec{b}) \times \vec{c} = \vec{a} \times \vec{c} + \vec{b} \times \vec{c}$
    
    $\vec{a} \times (\vec{b} + \vec{c}) = \vec{a} \times \vec{b} + \vec{a} \times \vec{c}$
\end{center}

\subsection{Записать формулу для вычисления векторного произведения в правом ортонормированном базисе.}

\begin{center}
$\vec{a} \times \vec{b} = 
\begin{vmatrix}
    \vec{i}&\vec{j}&\vec{k}\\
    a_1 & a_2 & a_3\\
    b_1 & b_2 & b_3
\end{vmatrix} = \vec{i}
\begin{vmatrix}
    a_2 & a_3\\
    b_2 & b_3
\end{vmatrix} - \vec{j}
\begin{vmatrix}
    a_1 & a_3\\
    b_1 & b_3
\end{vmatrix} + \vec{k}
\begin{vmatrix}
    a_1 & a_2\\
    b_1 & b_2
\end{vmatrix}, \vec{a}\{a_1;a_2;a_3\},\vec{b}\{b_1;b_2;b_3\}$
\end{center}

\subsection{Дать определение смешанного произведения векторов.}

Смешанным произведением 3 векторов $\vec{a}, \vec{b}, \vec{c}$ называется число, равное 
$(\vec{a} \times \vec{b}) \vec{c}$

Обозначение $(\vec{a}, \vec{b}, \vec{c})$, $\vec{a} \vec{b} \vec{c}$

\subsection{Сформулировать свойство перестановки (кососимметричности) смешанного произведения.}

При перестановке любых двух векторов, смешанное произведение меняет знак.

$\forall \vec{a}, \vec{b}, \vec{c}:\vec{a} \vec{b} \vec{c} = - 
\vec{b} \vec{a} \vec{c} = - \vec{c} \vec{b} \vec{a}$

\subsection{Сформулировать свойство линейности смешанного произведения.}

\begin{center}
$\forall \vec{a}, \vec{b}, \vec{c}, \forall k \in \mathbb{R}$

$(k\vec{a})\vec{b}\vec{c} = k(\vec{a}\vec{b}\vec{c})$

$\vec{a}(k\vec{b})\vec{c} = k(\vec{a}\vec{b}\vec{c})$

$\vec{a}\vec{b}(k\vec{c}) = k(\vec{a}\vec{b}\vec{c})$\\

$\forall \vec{a}, \vec{b}, \vec{c}, \vec{d}$

$(\vec{a}+\vec{d})\vec{b}\vec{c} = \vec{a}\vec{b}\vec{c} + \vec{d}\vec{b}\vec{c}$

$\vec{a}(\vec{b}+\vec{d})\vec{c} = \vec{a}\vec{b}\vec{c} + \vec{a}\vec{d}\vec{c}$

$\vec{a}\vec{b}(\vec{c}+\vec{d}) = \vec{a}\vec{b}\vec{c} + \vec{a}\vec{b}\vec{d}$
\end{center}

\subsection{Записать формулу для вычисления смешанного произведения в правом ортонормированном базисе.}

$$\vec{a}\vec{b}\vec{c} = 
\begin{vmatrix}
    a_1 & a_2 & a_3\\
    b_1 & b_2 & b_3\\
    c_1 & c_2 & c_3
\end{vmatrix} \ \vec{a}\{a_1;a_2;a_3\},
\vec{b}\{b_1;b_2;b_3\}, \vec{c}\{c_1;c_2;c_3\}$$

\subsection{Записать общее уравнение плоскости и уравнение «в отрезках». Объяснить 
геометрический смысл входящих в эти уравнения параметров.}



\subsection{Записать уравнение плоскости, проходящей через 3 данные точки.}



\subsection{Записать условия параллельности и перпендикулярности плоскостей.}



\subsection{Записать формулу для расстояния от точки до плоскости, заданной общим уравнением. }



\subsection{Записать канонические и параметрические уравнения прямой в пространстве. 
Объяснить геометрический смысл входящих в эти уравнения параметров.}



\subsection{Записать уравнение прямой, проходящей через две данные точки в пространстве. }



\subsection{Записать условие принадлежности двух прямых одной плоскости.}



\subsection{Записать формулу для расстояния от точки до прямой в пространстве.}



\subsection{Записать формулу для расстояния между скрещивающимися прямыми.}

\section{Теоретические вопросы повышенной сложности}

\subsection{Доказать геометрический критерий линейной зависимости трёх векторов.}

3 вектора линейно зависимы $iff$ они комплонарны

\begin{bf}Необходимость\end{bf}
Пусть $\vec{a}, \vec{b}, \vec{c}$ линейно зависимы, тогда один их них является линейной комбинацией
остальных


\subsection{Доказать теорему о разложении вектора по базису.}



\subsection{Доказать свойство линейности скалярного произведения.}



\subsection{Вывести формулу для вычисления скалярного произведения векторов, заданных в ортонормированном базисе.}



\subsection{Вывести формулу для вычисления векторного произведения в правом ортонормированном базисе.}



\subsection{Доказать свойство линейности смешанного произведения.}



\subsection{Вывести формулу для вычисления смешанного произведения трёх векторов в правом ортонормированном базисе.}



\subsection{Вывести формулу для расстояния от точки до плоскости, заданной общим уравнением. }



\subsection{Вывести формулу для расстояния от точки до прямой в пространстве.}



\subsection{Вывести формулу для расстояния между скрещивающимися прямыми.}


\end{document}